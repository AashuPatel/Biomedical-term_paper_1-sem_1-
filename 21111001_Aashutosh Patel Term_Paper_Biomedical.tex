\documentclass[12pt]{article}
\title{\Huge \textbf{Term Project}}
\usepackage{graphicx}
\usepackage{hyperref}
\graphicspath{{image/}}
\begin{document}
	\centering
	{\Huge“Cloud computing in Biomedical”}
	\vspace{4mm}
	\begin{figure}[h]
		\centering
		\includegraphics[scale=0.9]{nitlogo.png}
	\end{figure}
	
	\vspace{1cm}
	\begin{minipage}{0.5\textwidth}
		\begin{flushleft}
			\Large
			\underline{ Submitted By:}
			\newline
			Aashutosh Patel
			\\
			Roll No : 21111001
			\\
			Section: A “Biomedical”
			\\
			Semester: 1st
			\\
			Email id :\href{mailto:aashutoshpatelown@gmail.com}{aashutoshpatelown@gmail.com}
			\\
			Contact No:\href{tel:9044558649}{9044558649}
			
			
		\end{flushleft}
	\end{minipage}
	~
	\begin{minipage}{0.4\textwidth}
		\begin{flushright}
			\Large
			\underline{ Submitted To :}\\
			
			
			Dr. Saurabh Gupta \\ Department of Biomedical Engineering\\
			NIT Raipur,\\ Chattishgarh,\\
			India, 492013\\
			
			
			
			
		\end{flushright}
	\end{minipage}
	\clearpage 
	
	

	\section*{\Huge Acknowledgement}
	\raggedright
	\LARGE I am grateful to Dr. Saurabh Gupta for their proficient supervision of the term project on “Cloud computing in Biomedical”. I am very thankful to department for their continuous guidance and support.\\
	
	\raggedleft
	
	\vspace{5cm}
	
	\textbf{Aashutosh Patel}
	\\{Roll No:} 21111001
	
	
	1st semester,
	\\ Biomedical Engineering
	\\National Institute of Technology,Raipur
	\\
	\raggedright
	
	\vspace{3.5cm}
	Date of submission : 27march2022
	
	\tableofcontents
	\clearpage
	\section{Abstract}
	Healthcare studies are moving toward individualised measurement. There is need to 
	move beyond supervised assessments in the laboratory or clinic. Using good machinary to every clinic and laboratory  can provide a wealth of information on patient pathology and habitual behaviour, but cost and complexity of equipment have typically been a barrier. These cost and complexities can be removed with active use of remote computing. There has been a growing recognition in recent years of the usefulness of remote computing in medical research and diagnosis. It is likely that in time remote computers will effectively link medical facilities throughout the country and will service areas where adequate diagnostic facilities are not readily available.
	\\In this paper we will discuss the application and challenges of cloud computing in biomedical.
	\\
	\clearpage
	\section{Introduction}
	The informaton and communication technology today has enabled fast and rich way of exchanging information between people from diffrent domain with variety of informations. This advancement in technology can be used to process medical data remotely and also for remote laboratory where machinary with edge computing power is kept at one centralised place and researchers / students from diffrent organisations even if they are in remote areas.
	\\
	Real time monitoring of health parametres is possible with remote computers where we just take input signals from the patient and send it to centralised processing system which process data and upload redable output to interrnet which could be read from anywhere. The remote monitoring of the biomedical signal is an important tool for assessing the quality of life, control and prevention of diseases.
	\\
	\clearpage
	\section{Cloud Computing in Biomedical} 
	\begin{figure}[h]
	\centering	\includegraphics[scale=0.25]{img_01.jpg}
	\end{figure}

	Cloud computing can connect diffreant component of biomedical sector together with which each sector will be empoered with support of other sector like:
	\\
	i. Signals from paitient could be processed and sent to remote sitting doctor and he could check patient's condition.
	\\
	ii. Research laboratories could access clinical data and make better analysis report.
	\\
	iii. Remote location laboratories could perform complex calculation from a general computer of their lab.
	\\
	\clearpage
	\section{Need and benefits of Cloud Computing in Biomedical}
	We are entering into time where preference of personalised medication is growing more and more and for personalised medication we need more of input data from patient and its neither fesible nor practical to put complete dignostic machines to every house. For this purpose we could develop portable devices for taking vital parametres. These devises will just contain sensors and small communication module with which it will send data to remote computers via phones for data processing then send data to smartphone, with this we could reduce size of device as we are not using processing and display module. 
	\\
	\subsection{Benifits}
	Cloud computing is not just an alternative, it comes with many perks over in-lab computing.
	
	\subsubsection{Cost Efficient}
	As we are using centralised processor rather putting a processing unit with each sensor this reduces capital cost as well as maintainance cost.
	\\
	\subsubsection{Scalability}
	When the workload experiences significant change, a cloud can
	add or release resources in minutes. A cloud can provide extra processing resources during the peaks.
	\\
	\subsubsection{Superior Resiliency}
	Cloud vendors store backups of users’ applications and data in
	multiple geographical locations. If a machine fails, others can take
	over, at the same location, or between locations (for disaster
	recovery).
	\\
	\clearpage
	\section{Concerns and Scope of Improvement}
	As we are using clouds and servers, there is a possible risk of data leak which will expose many patients personal health information and this a matter of concern. In order to prevnet such unwanted incident there chould be strict government guidelines to follow while data exchange also a good and secure network should be built to access data.
	\\
	\vspace{4.5mm}
	\begin{figure}[h]
		\begin{center}
			\includegraphics[scale=0.7]{img_02.png}
		\end{center}
	\end{figure}
	
	To keep data protected new technology should be used like quantum and block chain technology.
	\\
	Also to improve accessibility and usability of remote computing, there is need of high speed internet network to remote areas.
	\\
	\section{Discussion and Conclusion}
	Use of cloud computing have already begun in biomedical domain but still it has huge subsectors to cover. Application of cloud computing will revolutionalize biomedical sector. 
	\\
	This technology comes with few concerning thing like risk of data breech, which should be restricted with advancement in technology and also with good governance.
	\\
	\newpage
	\section{References}
	
	i.\href{https://journals.plos.org/ploscompbiol/article?id=10.1371/journal.pcbi.1006144}{Cloud computing applications for biomedical
		science: A perspective
		-by Vivek Navale, Philip E. Bourne
		(Center for Information Technology, National Institutes of Health, Bethesda, Maryland, United States of America,  Department of Biomedical Engineering, University of Virginia, Charlottesville, Virginia, United States of America)}
	 
	\vspace{5mm}
	ii.\href{https://doi.org/10.1186/s12938-020-00825-9}{Towards remote healthcare monitoring 
		using accessible IoT technology: state‑of‑the‑art, insights and experimental design
		-by G. Coulby, A. Clear1, O. Jones, F. Young, S. Stuart and A. Godfrey}
		
	\vspace{5mm}
	iii. \url{https://www.researchgate.net/figure/Cloud-computing-paradigm-for-telemedicine_fig1_351607343}
	 
	
\end{document}